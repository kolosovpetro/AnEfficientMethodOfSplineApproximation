Consider the $m$-degree polynomial $P(m, X, N)$ having fixed non-negative integers $m$ and $N$
\begin{align*}
    P(m,X,N) = \sum_{r=0}^{m} \sum_{k=1}^{N} \coeffA{m}{r} k^r (X-k)^r
\end{align*}
For example
\begin{align*}
    P(2,X,0) &= 0 \\
    P(2,X,1) &= 30X^2 - 60X + 31 \\
    P(2,X,2) &= 150X^2 - 540X + 512 \\
    P(2,X,3) &= 420X^2 - 2160X + 2943 \\
    P(2,X,4) &= 900X^2 - 6000X + 10624
\end{align*}
where $\coeffA{m}{r}$ is a real coefficient defined recursively, see~\cite{alekseyev2018mathoverflow,
    on_the_link_between_binomial_theorem_and_discrete_convolution, unusual_identity_for_odd_powers,
    history_and_overview_of_polynomial_p}.
For example,
\input{sections/figures/05_fig_coefficients_a}

In this manuscript we discuss approximation properties of polynomial $P(m,X,N)$.
I use a few well-known criteria to measure and estimate error of approximation: Absolute error, Relative error and
Percentage error.
Assume that function $f_2(x)$ approximates the function $f_1 (x)$ then the errors are

\begin{align*}
    \mathrm{Absolute \; Error}   &= \frac{\lvert f_1(x) - f_2(x) \rvert}{\lvert f_1(x) \rvert} \\
    \mathrm{Relative \; Error}   &= \frac{\lvert f_1(x) - f_2(x) \rvert}{\lvert f_1(x) \rvert} \\
    \mathrm{Percentage \; Error} &= \frac{\lvert f_1(x) - f_2(x) \rvert}{\lvert f_1(x) \rvert} \times 100\%
\end{align*}

Diving straight to the point, we switch our focus to already mentioned polynomial $P(2,X,4) = 900X^2 - 6000X + 10624$
to show the first example of how it approximates the odd power function $X^5$.
In fact, we approximate the polynomial $X^{2m+1}$ by lower degree polynomial $X^m$ as the following image presents
\begin{figure}[H]
    \centering
    \includegraphics[width=1\textwidth]{sections/images/03_plots_polynomial_p2_n4_with_fifth}
    ~\caption{Polynomial plot $P(2, X, 4)$ with fifth power $X^5$.
    Points of intersection $X=4$, $X=4.42472$, $X=4.99181$.
    Interval of convergence: $3.9 \leq X \leq 5.3$ with $E < 2\%$.
    }\label{fig:03_plots_polynomial_p2_n4_with_fifth}
\end{figure}
As we see, the interval $3.9 \leq X \leq 5.3$ has the percentage error lesser than $2\%$ which is quite impressive.
Therefore, having fixed $N=4$ the polynomial $P(2, X, 4)$ approximates odd power in neighborhood of $N=4$
which is $3.9 \leq X \leq 5.3$.
To showcase the concrete values of absolute, relative and percentage errors of approximation above, I attach a separate
table in addenda.

One more interesting observation can be done by increasing the value of $N$ in $P(m, X, N)$ having fixed $m$, it
follows that by increasing $N$ the length of interval of convergence with odd-power $X^{2m+1}$ increasing as well.
For instance,
\begin{itemize}
    \item Having $P(2, X, 4)$ the interval of convergence with percentage error lesser than $1\%$ is $4.0 \leq X \leq 5.1$
    \item Having $P(2, X, 20)$ the interval of convergence with percentage error lesser than $1\%$ is $18.7 \leq X \leq 22.9$
    \item Having $P(2, X, 120)$ the interval of convergence with percentage error lesser than $1\%$ is $110.0 \leq X \leq 134.7$
\end{itemize}
The reason why the lenght of convergence interval rises as $N$ rise lays beneath the implicit form of polynomial $P(m,X,N)$
meaning that
\begin{align*}
    P(m,X,N) = \sum_{r=0}^{m} (-1)^{m-r} U(m, N, r) \cdot X^{r}
\end{align*}
where $U(m, N, r)$ is a polynomial defined as follows
\begin{align*}
    U(m, N, r) = (-1)^m \sum_{k=1}^{N} \sum_{j=r}^{m} \binom{j}{r} \coeffA{m}{j} k^{2j-r} (-1)^j
\end{align*}
which rises as $N$ rise.

To wrap up the current state of the manuscript, refresh the key facts and finding we got so far,
therefore, the polynomial $P(m,X,N)$ is an $m$-degree polynomial in $X$, having fixed non-negative
integers $m$ and $N$.
It approximates odd power function in some neighborhood of fixed $N$.
The length of interval of convergence between $X^{2m+1}$ and $P(m,X,N)$ rises as $N$ rise.

For the sake of clear and definite results verification I attach mathematica programs to generate
plots and data tables so that reader is able to verify the main results of current part of manuscript, these are
