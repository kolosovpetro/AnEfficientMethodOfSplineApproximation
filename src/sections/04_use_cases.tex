Use case scenarios of the approximation technique we discuss have their own constraints and limitations.
For instance, approximation requirements should have precisely specified exponent $j$ in $X^j$ because
for each $j$ there is a matching polynomial $P(m,X,N)$.
Perfectly, there should be a set precompiled polynomials $P(m,X,N)$ matching precise exponent $j$ in $X^j$ over
precisely defined approximation range with required error of approximation $E$ constraint.
Generally, approximation of power function $X^j$ by $P(m,X,N)$ can be split by the following steps
\begin{enumerate}
    \item Define exponent $j$ in $X^j$
    \item Define required error threshold $E$
    \item Define required interval of approximation $I$
    \item Choose and precompile polynomials $P(m,X,N)$
    such that required interval of approximation and error threshold $E$ are satisfied
    \item Define a set of knots so that error threshold $E$ and interval of approximation $I$ requirements are satisfied
\end{enumerate}
Defining set of spline knots essentially requires inspection of convergence intervals of between $X^j$ and $P(m,X,N+t_k)$
by choosing knots such that interval of approximation and error threshold are satisfied.
Consider an example.
Let be the following approximation requirement
\begin{enumerate}
    \item Exponent $j=3$
    \item Percentage error threshold $E\leq 1\%$
    \item Interval of approximation $10 \leq X \leq 15$
\end{enumerate}
Now we have to choose a set of polynomials $P(m, X, N+t_k)$ based on which we adjust a set of spline knots.
We can safely choose integers $t_k$ in range $10 \leq t_k \leq 15$ because
of the following properties of $P(m,X, N)$.
\begin{align*}
    P(m,X, X) &= X^{2m+1} \\
    P(m,X, X+1) &= (X+1)^{2m+1} - 1
\end{align*}
Therefore, for each two consequential points $N=X, N=X+1$ the absolute difference is 1, making that range
at least $1\%$ percentage error for $X^j \leq 100$.
I have chosen the approximation range $10 \leq X \leq 15$ and $j=3$ intentionally to show spline approximation with
percentage error lesser than $1\%$.
Therefore, to approximate $X^3$ in range $10 \leq X \leq 15$, we have to following spline function
\begin{align*}
    X^3 \approx
    \begin{cases}
        -2300 + 330X, & 10 \leq X < 11 \\
        -3025 + 396X, & 11 \leq X < 12 \\
        -3888 + 468X, & 12 \leq X < 13 \\
        -4901 + 546X, & 13 \leq X < 14 \\
        -6076 + 630X, & 14 \leq X \leq 15
    \end{cases}
\end{align*}
