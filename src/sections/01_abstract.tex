Let $P(m, X, N)$ be an $m$-degree polynomials in $X\in\mathbb{R}$
having fixed non-negative integers $m$ and $N$.
Essentially, the polynomial $P(m, X, N)$ is a result of a rearrangement inside Faulhaber's formula
in the context of Knuth's work entitled "Johann Faulhaber and sums of powers".
In this manuscript we discuss the approximation properties of polynomial $P(m,X,N)$.
In particular, the polynomial $P(m,X,N)$ approximates the odd power function $X^{2m+1}$ in a certain neighborhood
of a fixed non-negative integer $N$ with a percentage error less than $1\%$.
By increasing the value of $N$ the length of convergence interval with odd-power $X^{2m+1}$ also increases.
Furthermore, this approximation technique is generalized for arbitrary non-negative exponent of the power function $X^j$
by using splines.
