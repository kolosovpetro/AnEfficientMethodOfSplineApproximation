Let $P(m, X, N)$ be an $m$-degree polynomials in $X\in\mathbb{R}$
having fixed non-negative integers $m$ and $N$.
Essentially, the polynomial $P(m, X, N)$ is a result of rearrangement inside Faulhaber's formula
in context of Knuth's work \textit{Johann Faulhaber and sums of powers}.

In this manuscript we discuss approximation properties of polynomial $P(m,X,N)$.
In particular, the polynomial $P(m,X,N)$ approximates odd power function $X^{2m+1}$ in certain neighborhood
of fixed non-negative integer $N$ with percentage error lesser than $1\%$.

By increasing the value of $N$ the length of convergence interval with odd-power $X^{2m+1}$ increasing as well.

Furthermore, this approximation technique is generalized for arbitrary non-negative exponent power function $X^j$
by using splines.
