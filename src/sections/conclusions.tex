We have established that $P(m, X, N)$ is an $m$-degree polynomial in $X\in\mathbb{R}$
having fixed non-negative integers $m$ and $N$.

The polynomial $P(m, X, N)$ is a result of rearrangement inside Faulhaber's formula that
was inspired by Knuth's \textit{Johann Faulhaber and sums of powers}.

In this manuscript we have discussed approximation properties of polynomial $P(m,X,N)$.

In particular, the polynomial $P(m,X,N)$ approximates odd power function $X^{2m+1}$ in certain neighborhood
of fixed non-negative integer $N$ with percentage error lesser than $1\%$.

By increasing the value of $N$ the length of convergence interval with odd-power $X^{2m+1}$ increasing as well.

Furthermore, this approximation technique is generalized for arbitrary non-negative exponent power function $X^j$
by using splines.

In general, for each variation of $X^j$ such that $X^j \geq 100$ the approximation can be done using
splines in $P(m,X, N)$ over the interval $A \leq X \leq B$ with spline knot vector be the integers in
range from $A$ to $B$ so that knots vector is $\{A, A+1, A+2, \ldots, B \}$.
Because,
\begin{align*}
    P(m,X, X) &= X^{2m+1} \\
    P(m,X, X+1) &= (X+1)^{2m+1} - 1
\end{align*}
This can be further optimized depending on value of $N$ in $P(m,X,N)$ because convergence interval
with power function $X^j$ increases as $N$ rises.
