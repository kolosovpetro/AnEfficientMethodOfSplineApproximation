Previously, we have discussed that polynomial $P(m,X,N)$ approximates power function $X^j$
in some neighborhood of fixed non-negative integer $N$.
Approximation by $P(m,X,N)$ can be adjusted by $X^k$ multiplication for even exponents of power function.
In general, it is safe to say that power function $X^j$ is approximated by $P(m,X,N) \cdot X^k$
where $k=0$ for odd exponent $j$ and $k$ is either $k=1$ or $k=-1$ for an even exponent $j$.
Therefore, for arbitrary exponent $j$ in $X^j$ we have
\begin{align*}
    X^j \approx
    \begin{cases}
        P(m,X,N) \quad         & j=2m+1 \\
        P(m,X,N) \cdot X \quad & j=2m+2 \\
        P(m,X,N) \cdot X^{-1} \quad & j=2m \\
    \end{cases}
\end{align*}
Of course, there are other variations of the value of $k$, but we will stick to the simple case for the moment.

As we also discussed, the length $L$ of the convergence interval between $X^j$ and its approximation by $P(m,X,N)$
increases as $N$ grow.
However, the convergence interval is still bounded, which could not satisfy certain approximation scenarios.
Depending on the approximation requirements in terms of convergence interval length $L$ a single polynomial $P(m,X,N)$
with fixed $m$ and $N$ may be unsuitable.
Here is the place where spline approximation comes into play.
The spline $S(x)$ is piecewise defined function over the interval $(x_0, \ldots x_n)$
\begin{align*}
    S(x) &=
    \begin{cases}
        f_1(x), & x_0 \leq x < x_1 \\
        f_2(x), & x_0 \leq x < x_1\\
        \vdots & \vdots \\
        f_n(x), & x_{n-1} \leq x \leq x_n
    \end{cases}
\end{align*}
The given points $x_k$ are called \textit{knots}.

Assume that the approximation requirement in terms of convergence length $L$ is to approximate the power function $X^j$
bounded by real points $A$ and $B$ such that $A < B$.
Splines perfectly fit the need to match an arbitrary convergence range for the power function $X^j$ using the
approximation by $P(m,X,N)$.
Formally,
\begin{align*}
    X^j \approx
    \begin{cases}
        P(m,X,N+t_1) \cdot X^{k} \quad & x_0 \leq x < x_1 \\
        P(m,X,N+t_2) \cdot X^{k} \quad & x_0 \leq x < x_1 \\
        \vdots & \vdots \\
        P(m,X,N+t_{n-1}) \cdot X^{k} \quad & x_{n-1} \leq x < x_n
    \end{cases}
\end{align*}
The values of $t_r$ to be adjusted according to approximation requirements in terms of accuracy.
